\documentclass[tikz,border=1cm]{standalone}
\usetikzlibrary{matrix, shapes, arrows,calc, positioning}
\usetikzlibrary{decorations.pathreplacing}

\usepackage{amsmath}

\begin{document}    
%\begin{figure}
%\begin{center}
\tikzset{%
%blockA/.style = {rectangle, draw, text width=6cm, align=left, rounded corners, minimum height=6.2cm,minimum width=6cm},
%block/.style = {rectangle, draw, text width=6cm, align=left, rounded corners, minimum height=4em,minimum width=6cm},
%line/.style = {draw, -stealth},
wordBox/.style = {rectangle, draw, align=center, minimum height=3.5em, minimum width=4cm, text width=2cm},
stackedBox/.style = {wordBox, node distance=0pt, outer sep=0pt},
arrow/.style = {draw, -stealth}
}

\begin{tikzpicture}[auto]
%\matrix[matrix, column sep=2cm, row sep=2cm] (CPU_diagram){
\node [stackedBox] (graphBox1) {Graph 1};

\node [stackedBox, below=of graphBox1] (graphBox2) {Graph 2};
\node [stackedBox, below=of graphBox2] (graphBox3) {$\vdots$};
\node [stackedBox, below=of graphBox3] (graphBox4) {Graph $N$};
\node [below=of graphBox4] (MiniBatch) {\huge Mini-Batch};

\node [wordBox, above right=1cm and 3cm of graphBox1] (threadBox1) {Thread 1};
\node [wordBox, below=of threadBox1] (threadBox2) {Thread 2};
\node [wordBox, below=of threadBox2] (threadBox3) {$\vdots$};
\node [wordBox, below=of threadBox3] (threadBox4) {Thread 8};

\node [stackedBox, below right=1cm and 3cm of threadBox1] (gradBox1) {Gradient 1};
\node [stackedBox, below=of gradBox1] (gradBox2) {Gradient 2};
\node [stackedBox, below=of gradBox2] (gradBox3) {$\vdots$};
\node [stackedBox, below=of gradBox3] (gradBox4) {Gradient $N$};
 
\draw[
  decorate,
  decoration={brace, amplitude=10pt}
]
  ([xshift=4pt]gradBox1.north east) -- node (gradTip) {} ([xshift=4pt]gradBox4.south east) ;

\node [wordBox, right=2cm of gradTip] (SGD) {Stochastic Gradient Descent};

\foreach \i in {1, 2, 3, 4}{
    \path [arrow] (graphBox\i.east) -- (threadBox\i.west);
    \path [arrow] (threadBox\i.east) -- (gradBox\i.west);
}

\path [arrow] (gradTip) -- (SGD.west);

\end{tikzpicture}

% Place nodes with matrix nodes
% \matrix[matrix of nodes, column sep=1cm, row sep=1cm, below=1cm of CPU_diagram]{%
%\node [blockA] (n1) {Field Tapes and observers log\\
%(1) Pre processing\\
%\begin{itemize}
%\item De .... can not read
%\item Reformatting
%\item Editting
%\item Geometric spreading
%\item Setup of field Geometry
%\item Application of field statics
%\end{itemize}};
%& \node [block] (n8) {DMO Correction};
%\node [block,below=1cm of n8] (n9) {Inverse NMO Correction};
%\node [block,below=1cm of n9] (n10) {Velocity Analysis}; \\
%\node [block] (n2) {(2) Deconvolution and Trace };   
%&\node [block] (n11) {NMO correction};\\
%\node [block] (n3) {(3) cmp sorting};                 
%& \node [block] (n12) {Deconvolution};\\
%\node [block] (n4) {(4) velocity analysis};              
%& \node [block] (n13) {Time-variant spectrum Whitening};   \\
%\node [block] (n5) {(5) Residual statics corrections}; 
%&\node [block] (n14) {Time-variant Filtering};     \\
%\node [block] (n6) {(6) Velocity Analysis};                                          
%& \node [block] (n15) {Migration};   \\
%\node [block] (n7) {(7) NMO Correction};                                           
%& \node [block] (n16) {Gain Application};\\
%};
%% Draw edges
%\foreach \i/\j in {1/2,2/3,3/4,4/5,5/6,6/7}{
%  \path [line] (n\i) -- (n\j);}
%
%\foreach \i/\j in {8/9,9/10,10/11,11/12,12/13,13/14,14/15,15/16}
%    \path [line] (n\i) -- (n\j);
%
%\path [line] (n7.east) -- ++(0.5cm,0)coordinate(a) -- (a |- n8.north)-- ++(0,0.5cm) -| (n8.north);

%\end{center}
%\end{figure}
\end{document}
